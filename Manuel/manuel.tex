%this file is the IVM article analysis
%a % comment anything after % until the end of the line
%minimum references to begin our article
\documentclass[18pt]{article}
\let\myBib\thebibliography
\let\endmyBib\endthebibliography
\renewcommand\thebibliography[1]{\ifx\relax#1\relax\else\myBib{#1}\fi}
\usepackage[english]{babel}
\usepackage[T1]{fontenc}
\usepackage{graphicx}
\usepackage{hyperref}
\usepackage{float}
\usepackage{amsmath}
\usepackage[margin=1in]{geometry}
\usepackage{indentfirst}
\usepackage{titlesec}
\newcommand{\sectionbreak}{\clearpage}

%presentation of the document
\title{Manuel : As if the world wasn't ending}
\author{Mika�l \textsc{Demirdelen}\\ Olivir \textsc{Frischo}\\ Julien \textsc{Bouvet}}
\date{14/11/2015}
\setlength\parindent{50pt}
\begin{document}

\maketitle

\section{Introduction}
Cette section explique le principe du jeu. 
En gros, on poss�de un groupe de gens qu'on doit faire aller d'un point d'une map � un autre.
\subsection{D�roulement d'une partie}
Au d�but, chaque personnage se trouve sur une partie de la Tile de commencement. A chaque tour, chaque personnage a le droit de faire une action. Cette action peut �tre :
\begin{itemize}
  \item Se reposer.
  \item Se cacher.
  \item Parler avec.
  \item Se d�placer.
  \item Crier.
  \item ...
\end{itemize}
Le joueur d�cide de mani�re s�quentielle chaque action faite par ses personnages. Ensuite, pendant le tour des zombis, ceux-ci font automatiquement des actions selon celles-ci :
personnage a le droit de faire une action. Cette action peut �tre :
\begin{itemize}
  \item Ne rien faire.
  \item Se diriger vers le bruit.
  \item Se diriger vers la lumi�re (pour les �gouts ou la nuit).
  \item ...
\end{itemize}

\section{Personnages}
\subsection{Caract�ristiques}
\subsection{Classes de personnages}
\subsection{Personnages de commencement}
\subsection{PNJ}
\subsubsection{PNJ recrutables}
\subsubsection{PNJ mauvais}

\section{Zombis}
\subsection{Caract�ristiques des zombis}
\subsection{Type d'action face aux zombis}
Il y aura 2 types d'actions suppl�mentaire qu'un personnage peut faire s'il se trouve sur la m�me partie de tile qu'un zombi :
\begin{itemize}
  \item Fuir.
  \item Attaquer.
\end{itemize}
\subsubsection{Points de fatigue}
A chaque action, le personnage faisant cette action gagne un point de fatigue.
Un personnage perds ses points de fatigue en ne faisant rien.
\subsection{Types de zombis}
Il y aura 5 types de zombis :
\begin{itemize}
  \item Les zombis faibles : ils sont simples � vaincre mais sont forts en horde.
  \item Les zombis rapides : ils sont simples � vaincre mais sont difficiles � fuir.
  \item Les zombis forts : ils sont difficiles � vaincre mais peuvent �tre fuit facilement.
  \item Les zombis infect�s : ils ont les m�me stats que les zombis faibles mais transforment en zombis directement.
  \item Les zombis boss : ils sont forts et rapides, mais ne sont jamais en horde.
\end{itemize}
\section{Combats}
\subsection{Attaque par le joueur}
\subsection{Attaque par le zombi}

\section{Map}
Une map est compos�e de tiles carr�es. Chaque tile poss�de 5 parties : Centre, Nord, Sud, Est et Ouest.
\subsection{Types de tiles et sorties}
Il y aura XX types de maps :
\begin{itemize}
  \item Street : Les sorties se font sur les cot�s oppos�s (Nord -> Sud).
  \item Building : Les sorties se font sur le m�me c�t� (Nord -> Nord).
  \item Egouts : Certaines parties de la tile ne seront pas praticables (d� � l'eau).
\end{itemize}
\subsection{Tailles de maps}
Il y aura 3 types de maps :
\begin{itemize}
  \item Petite : Les maps de d�but de jeu. Elles feront 8x8 tiles. Elles ne comporteront qu'un seul niveau.
  \item Moyenne : Les maps feront 15x15 tiles. Elles comporteront deux niveaux.
  \item Grande : Les maps feront 20x20 tiles. Elles comporteront trois niveaux ou plus.
\end{itemize}
\subsection{Interaction entre la map et les personnages}




\bibliography{bibliography}
\bibliographystyle{plain}

\end{document}
