\section{Zombis}
\subsection{Types de zombis}
Il y aura 5 types de zombis :
\begin{itemize}
  \item Les zombis faibles : ils sont simples à vaincre mais sont forts en horde.
  \item Les zombis rapides : ils sont simples à vaincre mais sont difficiles à fuir.
  \item Les zombis forts : ils sont difficiles à vaincre mais peuvent être fuit facilement.
  \item Les zombis infectés : ils ont les même stats que les zombis faibles mais transforment en zombis directement.
  \item Les zombis boss : ils sont forts et rapides, mais ne sont jamais en horde.
\end{itemize}
\subsection{Caractéristiques des zombis}
Les zombis aussi possèdent des caractéristiques propres :
\begin{itemize}
  \item Type : Le zombi possède un type correspondant à ceux détaillés plus haut.
  \item Vie : Le zombi possède un certain nombre de point de vie en fonction de son type. Lorsqu'il perd tous ses points de vie, le zombi meurt.
  \item Attaque.
  \item Précision.
  \item Point de mouvement.
  \item Vision : Cet attribut correspond à la portée à laquelle il peut voir les personnages.
  \item Ouïe : Cet attribut correspond à la portée à laquelle il peut entendre les personnages.
\end{itemize}
En fonction de son type le zombi aura des caractéristiques de base différentes.
\subsubsection{Zombis faibles}
\begin{itemize}
  \item Vie : 2.
  \item Attaque : 1.
  \item Précision : 3.
  \item Point de mouvement : 1.
  \item Vision : 5.
  \item Ouïe : 5.
\end{itemize}
Ce type de zombi possède des caractéristiques pouvant varier de +/- 10\%. \textbf{(à voir dans l'équilibrage si, lorsqu'on a un nombre à virgule, on l'arrondi ou on le tronque)}
\subsubsection{Zombis rapides}
\begin{itemize}
  \item Vie : 2.
  \item Attaque : 1.
  \item Précision : 5.
  \item Point de mouvement : 2.
  \item Vision : 5.
  \item Ouïe : 5.
\end{itemize}
Ce type de zombi possède des caractéristiques pouvant varier de +/- 10\%. \textbf{(à voir dans l'équilibrage si, lorsqu'on a un nombre à virgule, on l'arrondi ou on le tronque)}
\subsubsection{Zombis forts}
\begin{itemize}
  \item Vie : 4.
  \item Attaque : 3.
  \item Précision : 3.
  \item Point de mouvement : 1.
  \item Vision : 5.
  \item Ouïe : 5.
\end{itemize}
Ce type de zombi possède des caractéristiques pouvant varier de +/- 10\%. \textbf{(à voir dans l'équilibrage si, lorsqu'on a un nombre à virgule, on l'arrondi ou on le tronque)}
\subsubsection{Zombis boss}
\begin{itemize}
  \item Vie : 8.
  \item Attaque : 4.
  \item Précision : 7.
  \item Point de mouvement : 2.
  \item Vision : 10.
  \item Ouïe : 10.
\end{itemize}
Ce type de zombi possède des caractéristiques pouvant varier de +/- 20\%. \textbf{(à voir dans l'équilibrage si, lorsqu'on a un nombre à virgule, on l'arrondi ou on le tronque)}
\subsection{Génération de zombis}
Sur chaque carte, il y a des points de génération de zombis. Ceux-ci possèdent une portée nommée p et un seuil s et une fréquence d'apparition f. La portée se compte en terme de changement de tile sur les directions verticales et horizontales. Considérons N le nombre de zombis dans la zone couverte par p, et P le nombre de tiles couvertes par la portée p. Considérons aussi les fonctions Arr(x) qui arrondi x à l'inférieur, et Pos(x) qui rends x s'il est supérieur à zéro.
\\\\
A la fin de chaque tour (humain+zombi), il apparait à ce point de génération un nombre de zombis, noté Z, selon cet algorithme :\\\\
Z = Arr[Pos((s-N)*(f-(f*N/2*P))].
\textbf{Formule provisoire.}\\\\
Dans cette formule, le premier terme permet simplement d'empêcher la génération de zombis s'il y a au sein de la zone couverte par la portée, un nombre de zombi supérieur au seuil.
