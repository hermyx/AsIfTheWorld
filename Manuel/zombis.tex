\section{Zombis}
\subsection{Caractéristiques des zombis}
\subsection{Type d'action face aux zombis}
Il y aura 2 types d'actions supplémentaire qu'un personnage peut faire s'il se trouve sur la même partie de tile qu'un zombi :
\begin{itemize}
  \item Fuir.
  \item Attaquer.
\end{itemize}
\subsubsection{Points de fatigue}
A chaque action, le personnage faisant cette action gagne un point de fatigue.
Un personnage perds ses points de fatigue en ne faisant rien.
\subsection{Types de zombis}
Il y aura 5 types de zombis :
\begin{itemize}
  \item Les zombis faibles : ils sont simples à vaincre mais sont forts en horde.
  \item Les zombis rapides : ils sont simples à vaincre mais sont difficiles à fuir.
  \item Les zombis forts : ils sont difficiles à vaincre mais peuvent être fuit facilement.
  \item Les zombis infectés : ils ont les même stats que les zombis faibles mais transforment en zombis directement.
  \item Les zombis boss : ils sont forts et rapides, mais ne sont jamais en horde.
\end{itemize}
\subsection{Génération de zombis}
Sur chaque carte, il y a des points de génération de zombis. Ceux-ci possèdent une portée nommée p et un seuil s et une fréquence d'apparition f. La portée se compte en terme de changement de tile sur les directions verticales et horizontales. Considérons N le nombre de zombis dans la zone couverte par p, et P le nombre de tiles couvertes par la portée p. Considérons aussi les fonctions Arr(x) qui arrondi x à l'inférieur, et Pos(x) qui rends x s'il est supérieur à zéro.
\\\\
A la fin de chaque tour (humain+zombi), il apparait à ce point de génération un nombre de zombis, noté Z, selon cet algorithme :\\\\
Z = Arr[Pos((s-N)*(f-(f*N/2*P))].
\textbf{Formule provisoire.}\\\\
Dans cette formule, le premier terme permet simplement d'empêcher la génération de zombis s'il y a au sein de la zone couverte par la portée, un nombre de zombi supérieur au seuil.
