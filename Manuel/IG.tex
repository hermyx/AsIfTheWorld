\section{Interface Graphique}
\subsection{Vue générale}
\subsection{Données de l'ATH}
\subsection{Caractéristiques des personnages}
Il faut que lorsque le joueur sélectionne un de ses personnage, il puisse voir, en priorité les propriétés dynamiques de celui-ci. C'est-à-dire son stress, sa santé mentale, sa fatigue et son bruit. En faisant une manipulation spéciale (clic droit, touche clavier, clic sur menu ou icône, ...), le joueur doit pouvoir avoir accès à toutes les capacité de son personnage. Que ce soit sur l'ATH ou sur le menu de capacités, les valeurs numériques ne sont pas données : Seul un mot variant de manière "numérique" donne une indication au joueur de la valeur de l'attribut. Cette variation se fait de manière proportionnelle et pas absolue. Le mot peut être superposé à un cadre coloré donnant une indication "analogique" de la valeur, allant de blanc à noir, ce dernier étant lorsque la valeur est à l'extrémité la plus défavorable pour le joueur.
\\\\
Par exemple, lorsque le joueur sélectionne un personnage, si celui-ci a 20 de stress, l'ATH indique qu'il a un stress "Faible" sur fond légèrement gris. S'il a 50 de stress, il a un stress "Moyen" sur gris moyen. Et s'il a 80 de stress, le stress est "Fort" sur fond gris foncé. S'il monte à 99 de stress, il est toujours noté "Fort" mais sur un fond quasiment indissociable du Noir.
\\\\
Sur la fiche descriptive générale du personnage (le menu avec les caractéristiques) et au moins pour le mode Histoire, chaque personnage de début doit avoir son histoire personnelle d'écrite afin que le joueur cerne au mieux la psychologie des personnages.
\\\\
Lorsque le stress, la fatigue et la santé mentale atteignent leurs valeurs maximales (minimales pour la santé mentale), il faut que le joueur en soit averti. Le texte descriptif du personnage doit alors être modifié (ou remplacé) par un texte expliquant la réaction du personnage. Par exemple, s'il n'a plus de santé mentale et qu'il entre dans l'état de schizophrénie N.2, le joueur doit alors avoir un texte décrivant globalement le fait que son comportement ne soit plus entièrement ni raisonné, ni raisonnable et que sa situation semble temporaire.
\subsection{Relations entre les personnages}
Possibilité de bloquer les relations
\subsection{Actions}
Le menu d'action est composé une barre de rechercher et d'un texte indiquant la manipulation pour afficher la liste des actions.