\section{Actions}
\subsection{Type d'actions}
\subsection{Caractéristiques des actions}
Une action possède plusieurs caractéristiques :
\begin{itemize}
  \item Nom : Correspond à un bref descriptif de l'action que le joueur verra et utilisera.
  \item Condition : Correspond à la condition nécessaire à effectuer l'action. Par exemple, la condition de l'action "Attaquer" est d'être à la bonne portée de la cible.
  \item Réussite : Correspond à la condition de réussite d'une action. Par exemple, pour l'action "Attaquer", cela correspond à la condition de Chance (voir section Combat). Un autre exemple, pour immobiliser un autre personnage, il faut que la force+constitution du personnage soit supérieur à la force+constitution de sa cible.
  \item Choc : Capacité d'un action à choquer les personnes voyant cette action. Elle correspond à une valeur numérique comprise entre 0 et 5.
  \item Fatigue : Capacité d'une action à fatiguer ceux qui la réalisent. Elle peut être comprise entre 0 et 5.
\end{itemize}
\subsection{Liste des actions possibles}


Notes : Pour le déplacement, intégrer la fatigue dans la formule !!!